\documentclass[../main.tex]{subfiles}

\begin{document}

\chapter*{Appendices} \label{chapter:appendices}

\addcontentsline{toc}{chapter}{Appendices}
\renewcommand{\thesection}{\Alph{section}}
\setcounter{section}{0}
\renewcommand*{\theHsection}{chX.\the\value{section}}

\section{Abbreviations} \label{appendix:abbreviations}

\appendixtable{l l}{
    UAV & Unmanned aerial vehicle \\
    ELMO & Electric modular \\
    GDP & Group design project \\
    ESC & Electronic speed controller \\
    EDMC & Engineering design and manufacturing workshop \\
    TSRL & Testing and structures research laboratory \\
    CNC & Computer numerical control \\
    PUC & Power unit cell \\
    CG & Centre of gravity \\
    LE & Leading edge \\
    AC & Aerodynamic centre \\
    IMU & Inertial measurement unit \\
    HLD & High lift device \\
}

\section{Parameters} \label{appendix:parameters}

\appendixtable{l l}{
    $A$ & Aspect ratio \\
    $\alpha$ & Angle of attack, normally defined relative to the zero lift line \\
    $b$ & Span \\
    $c$ & Chord length \\
    $C_\mathrm{D}$ & Drag coefficient \\
    $C_\mathrm{L}$ & Lift coefficient \\
    $C_\mathrm{M}$ & Moment coefficient \\
    $C\mathrm{n}$ & Control derivative coefficient \\
    $\delta$ & Control surface deflection \\
    $e$ & Oswald span efficiency \\
    $\epsilon$ & Downwash \\
    $\eta$ & Efficiency \\
    $F$ & Frictional force \\
    $g$ & Gravitational acceleration \\
    $h$ & Distance between LE and CG as a proportion of $c$ \\
    $h_0$ & Distance between AC and CG as a proportion of $c$ \\
    $I$ & Mass moment of inertia \\
    $K$ & Induced drag factor \\
    $l$ & Moment arm length \\
    $L$ & Lift force \\
    $m$ & Mass \\
    $M$ & Moment \\
    $n$ & Load factor \\
    $P$ & Roll rate \\
    $\phi$ & Bank angle \\
    $q$ & Dynamic pressure \\
    $\rho$ & Density of air \\
    $S$ & Planform area \\
    $T$ & Thrust force \\
    $t$ & Time \\
    $\tau$ & Angle of attack effectiveness \\
    $\theta$ & Generic angle \\
    $V$ & Velocity \\
    $\overline{V}$ & Volume coefficient \\
    $W$ & Weight \\
    $w$ & Height \\
    $x$ & Roll axis \\
    $y$ & Pitch axis \\
    $z$ & Yaw axis \\
    $\lambda$ & Taper ratio \\
}

\section{Subscripts} \label{appendix:subscripts}

\appendixtable{l l}{
    0 & Relating to the zero lift condition \\
    0L & Relating to the zero lift line \\
    a & Relating to aileron control surfaces \\
    AC & Relating to the AC of the lifting surface \\
    approach & Relating to the approach of the aircraft \\
    c & Relating to the aircraft in cruise \\
    D & Relating to drag value or centre of action \\
    e & Relating to elevator control surfaces \\
    f & Relating to flap HLDs \\
    F & Relating to the fuselage \\
    h & Relating to the horizontal stabiliser \\
    i & Relating to the inboard location of a surface \\
    L & Relating to the lift of a surface \\
    M & Relating to the pitch moment of the aircraft \\
    MAC & Related to the mean aerodynamic chord of a surface \\
    mg & Relating to the main gear \\
    n & Relating to the control derivative of a surface \\
    NoHLDs & Relating to the surface area of surface with no trailing edge taken up by control surfaces or HLDs \\
    o & Relating to the outboard location of a surface \\
    r & Relating to the rudder \\
    SS & In a steady state condition \\
    T & Relating to the thrust or thrust centre of action \\
    TO & Relating to the takeoff condition \\
    v & Relating to the vertical stabiliser \\
    W or none & Relating to the wing \\
    WF & Relating to the combined wing and fuselage \\
    WT & Relating to the combined wing and tail \\
    yy & Relating to action about the pitch axis \\
    zz & Relating to axis about the yaw axis \\
    $\alpha$ & Denotes a derivative wrt. angle of attack \\
    $\delta$ & Denotes a derivative wrt. a control surface deflection \\
}

\section{Calculation process for elevator deflection angles} \label{appendix:elevator-deflection}

Rearrangement process to determine elevator deflection angles provided by Prof. Towell (Towell, 2019):

Beginning with a moment balance about the centre of gravity:

\importequation{
	L(h-h_0)c+Tz_T+M=L_hl	
	}{moment-balance-towell}

Which becomes:
	
\importequation{
	qSC_L(h-h_0)c+Tz_T+qScC_{M_0}=qS_hlC_{L_h}
}{moment-balance-rearranged-towell}
	
Rearrange for $C_{L_h}$:

\importequation{
	C_{L_h}=\frac{qSC_L(h-h_0)c+Tz_T+qScC_{M_0}
	}{qS_hl}
}{CLT_rearrange-towell}
	
Then performing analysis of vertical equilibrium:

\importequation{
	l+L_h=nW
}{vert-force-balance-towell}
	
Rearranging: 

\importequation{
q(C_LS+S_hC_{L_h})=nmg
}{vert-balance-rearranged-towell}
	
Substituting for $C_{L_h}$ and simplifying:

\importequation{
	qSC_L+\frac{qSC_L(h-h_0)c+Tz_T+qScC_{M_0}}{l}=nmg
}{substitution-towell}

Rearranging for q:

\importequation{
	q=\frac{nmgl-Tz_T}{SC_Ll+SC_L(h-h_0)c+ScC_{M_0}}
}{q-towell}
	
Now since $q=1/2\rho V^2$, and $C_L=C_{L_\alpha}  \alpha$, this can be substituted for q and $C_L$, rearranged for $\alpha$ and simplified (n excluded as flying in cruise condition, so n = 1):

\importequation{
	\alpha = \frac{2(mgl-Tz_T)V^{-2}-cC_{M_0}}{\rho S C_{L_\alpha}(h-h_0)c}
}{alpha-towell}

With $\alpha$ found, it can be used to find the tail lift coefficient for that particular speed:

\importequation{
	C_{L_h}=\frac{qSC_L(h-h_0)c+Tz_T +qScC_{M_0}}{qS_hl}
}{new-CLh-towell}

\importequation{
	=\frac{C_{L_\alpha}\alpha(h-h_0)+\frac{Tz_T}{qSc}+C_{M_0}}{\frac{S_hl}{SC}}
}{clt-rearrangement-2-towell}
	   	
Now $\frac{S_h  l}{S c}=\bar{V}_h$, so this becomes (final form of $C_{L_h}$):

\importequation{
	C_{L_h}=\frac{C_{L_\alpha}\alpha(h-h_0)+\frac{Tz_T}{qSc}+C_{M_0}}{\bar{V}_h}
}{Final-clt-towell}
	
With this the elevator deflection can finally be calculated using:

\importequation{
	\delta =\frac{C_{L_h}}{C_{L_{{\alpha}_h}}}-\alpha (1-{\epsilon}_\alpha)-{\epsilon}_0-\frac{{\alpha}_h}{{\tau}_e}
}{elevator-deflection-towell}
	
This can then be converted to degrees.

\section{Calculation process for initial parameter selection} \label{appendix:calculation-process-for-initial-parameter-selection}

Motor, battery, and ESC selection:

\appendiximage{motor-battery-esc-selection-table}{0.95}

Propeller selection:

\appendiximage{propeller-selection-table}{0.95}

\section{Initial project plan} \label{appendix:initial-project-plan}

\appendiximage{initial-project-plan}{0.95}

\section{Uncertainty analysis tables} \label{appendix:uncertainty-analysis-tables}

Uncertainty calculation table for \cd:

\appendiximage{uncertainty-table-a}{0.95}

Uncertainty calculation table for lift:

\appendiximage{uncertainty-table-b}{0.95}

Uncertainty calculation table for $C_\mathrm{PM}$:

\appendiximage{uncertainty-table-c}{0.95}

\section{Initial constraint analysis} \label{appendix:initial-constraint-analysis}

Function used to form an early estimate of the cruise velocity:

\begin{lstlisting}[language=python,firstnumber=1]
import numpy as np


def v_cruise():
    mass_kg = 7
    lift_n = 9.81 * mass_kg
    cl = np.random.normal(0.8, 0.1)
    ar = np.random.normal(10, 1)
    b = np.random.normal(1.8, 0.2)
    s = (b ** 2) / ar
    return np.sqrt(lift_n / (0.5 * 1.225 * s * cl))
\end{lstlisting}

Function used to form an early estimate of the cruise power:

\begin{lstlisting}[language=python,firstnumber=1]
import numpy as np


def power():
    efficiency = np.random.normal(0.6, 0.1)
    mass_kg = 7
    lift_n = 9.81 * mass_kg
    v_cruise = np.random.normal(21, 0.5)
    cl = np.random.normal(0.8, 0.1)
    s = lift_n / (0.5 * 1.225 * (v_cruise ** 2) * cl)
    cd = np.random.normal(0.1, 0.02)
    drag = 0.5 * 1.225 * (v_cruise ** 2) * s * cd
    return drag * (v_cruise / efficiency)
\end{lstlisting}

\section{XFOIL code for hinge moment determination} \label{appendix:xfoil-code}

\begin{lstlisting}[language=python,firstnumber=1]
XFOIL c> NACA6412 (for wing, NACA0018 for tail surfaces) 
XFOIL c> GDES 
.GDES c> flap 
Enter flap hinge x location r> [hinge location value] 
Enter flap hinge y location (or 999 to specify y/t) r> 999 
Enter flap hinge relative y/t location r> 0.5 
Enter flap deflection in degrees (+ down) r> [flap or control surface deflection value] 
.GDES c> eXec 
.GDES c> {ENTER} 
XFOIL c> oper 
.OPERi c> r [Reynolds number value] 
.OPERi c> m [Mach number value] 
.OPERi c> visc 
.OPERv c> iter [iteration count, 250 used] 
.OPERv c> alfa [lifting surface incidence] 
.OPERv c> fmom 
\end{lstlisting}

\end{document}
