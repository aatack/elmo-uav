\documentclass[../../main.tex]{subfiles}

\begin{document}

\newchapter{Design Brief}{design-brief}

\section{Project objectives} \label{section:design-brief:project-objectives}

This project aimed to create a configurable, stable and reliable test platform for determining the effect of the position of propulsion systems on the efficiency of an electric aerial vehicle.
Specific emphasis was placed on isolating the effect of the location of the propellers on the power required to sustain flight at a particular velocity, through mechanisms such as promoting airflow over lifting surfaces or counteracting wingtip vortices, while negating the effect of other variables such as the mass or frontal area by keeping them constant. 

Integral to the success of this project was the construction of a robust UAV, representing a typical and generic aircraft and as such capable of producing results that are widely applicable, but which can have propulsion units fixed to it in an entirely modular fashion.
This allows the construction of a wide range of flight configurations, altering the characteristics of the propulsion system by changing its position between the front and rear of the fuselage; multiple locations along the span of the wing, including the wingtips; above or below the wing; or in a tractor or pusher configuration. 

By constructing a model capable of completely traversing the design space, a number of experiments were to be run which, when combined with results of computational experiments, would yield insight as to the optimal propulsion configuration for similar fixed-wing electric vehicles. 

\section{Stakeholder analysis} \label{section:design-brief:stakeholder-analysis}

\importimage{stakeholder-analysis}{summary of invested partners, members, and consumers.}{Stakeholder analysis}{0.9}

The aim of this project is to create a test craft which can be used to find the most efficient configuration of electric propulsion units for the use of electric propulsion in commercial aircrafts in the future.
This project was proposed by Primary stakeholder and Southampton alumni Dr Ewan Kirk, whose vision for the electrification of commercial air travel was the driving force behind the project.
Climate change affects everyone around the world, with the issue only worsening at present.
The airline industry is a large contributor to the issue, so the eventual shift towards electric propulsion is becoming more and more of a necessity.
The proposition of this project and the investment provided by Dr Ewan Kirk is one of many steps on the path to an all-electric future in the commercial airline industry.  

The secondary stakeholder is the University of Southampton.
As a project associated to the university, the outcome is important to the profile and stature of the university as one of the leading Engineering Universities in the country. 

The second set of stakeholders are the employees of a company.
In this case the ‘employees’ are the members of GDP 47 as well as the supervisor Prof. Keith Towell, and co-supervisor Dr Mario Ferraro. 

At the moment the main stakeholders are those stated above.
However, the shift to electric propulsion is not a new concept in the airline industry, with companies looking into its possibility and feasibility.
As a result, in the future new stakeholders may appear, such as aircraft and engine manufacturers who intend to implement electronic propulsion. 

\section{Resources} \label{section:design-brief:resources}

Like all GDP teams, the project had an initial project allocation of £850 to design and build the UAV.
As the project had been proposed by alumni, Ewan Kirk, he had pledged an additional £1500 towards the development if it were required, however this was to be used as a last resort.
Finally, through a successful pitch to the Northrop Grumman 3D printing fund, our team leader managed to secure an additional up to £1000 allocation to be spent on 3D printed parts. 

The team also had access to university resources provided by our course, which included access to workshops such as the EDMC and TSRL, and specialist tools such as metalworking machinery, a CNC laser cutter, a CNC hot wire foam cutter, and the university 3D printers.
We also had access to electronics workshops and equipment, as well as general access to vital tools and fasteners.

\end{document}
