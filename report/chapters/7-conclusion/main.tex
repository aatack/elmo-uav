\documentclass[../../main.tex]{subfiles}

\begin{document}

\newchapter{Conclusion}{conclusion}

This project intended to identify the most efficient power unit location on the body of a conventional airframe through the flight testing of a UAV with a movable propulsion unit.
Despite the steady design progress made by all team members, the manufacture and flight testing of the UAV was not possible due to the COVID-19 lockdown restrictions, however, the project was on course to achieve this and complete the desired outcomes.  

The design has been shown to meet the requirements set out by the stakeholder, and the design specification.
The model incorporates reconfigurable propulsion units on the wings and fuselage to allow the testing of multiple propulsion unit locations.
This allows for an informed decision on the most efficient configuration.
An early design was tested both physically in the wind tunnel and mathematically.
The final design was tested in CFD, FEA and mathematically to confirm its airworthiness, allowing for the confidence that the proposed design would be a suitable, stable flight platform for the necessary tests.
The construction of the UAV was planned to be simple to allow for fast changes of the power unit configuration as specified by the hardware requirements. 

Thorough testing of the electrical systems showed that the UAV would be fully controllable by a pilot, whilst recording the necessary data during the flights to determine the most efficient propulsion unit location.
Unfortunately, the autopilot functionality could not be tested due to the lockdown.  

The vision for this project does not end with the conclusion of our allocated GDP time.
The goal of Dr. Ewan Kirk was to obtain a model UAV that would function as a test bed to provide research for future electrical aircraft development.
There is significant work that can be done in the future on this project with the completion of the manufacture and testing of the design laid out in this report, followed by the refinement of a maximum efficiency design based on the results of those tests. 

\end{document}
