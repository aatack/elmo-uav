\documentclass[../../main.tex]{subfiles}

\begin{document}

\chapter{Introduction} \label{chapter:introduction}

\section{Motivation} \label{sec:introduction:motivation}

While air travel is hugely popular and brings enormous economic benefits, it is undeniable that the continued use of fossil fuels in such a demanding market has negative and unsustainable effects on the planet.
By removing some limitations of conventional aircraft, electric aircraft may one day supercede aircraft powered by fossil fuels.
That is the vision of University of Southampton alumnus Dr. Ewan Kirk, who initiated this project in the hopes of accelerating that trend.
Given the aviation industry's apparent lack of progress in this field, a potentially world-changing market opportunity could be opening.

The commercial aviation industry is lagging on the shift toward electric aviation due to the hardware and power density limitations of current batteries.
By optimising every aspect of an aircraft's design, perhaps outside the norm seen in conventional aircraft, electrically driven passenger and commercial aircraft may one day become a reality rather than a gimmick.
Efficiency is paramount in electrical aircraft.
This project hopes to contribute towards the shift to commercial electric aircraft by providing a highly modular test platform for researching the most efficient location for electric propulsion units. 

Throughout this report, the entire process of the project's development will be documented in five main areas.

\begin{itemize}
    \item Why is a project like this beneficial, and what impact could it potentially have?
    \item How do the project objectives translate into engineering criteria?
    \item Which process was followed to produce a design meeting these criteria?
    \item What was the outcome of the design process?
    \item Was the project successful, and how could its outputs be further developed going forwards?
\end{itemize}

\section{Background research} \label{sec:introduction:background-research}

\subsection{High-level design} \label{sec:introduction:background-research:high-level-design}

% Small Unmanned Fixed-Wing Aircraft Design: A Practical Approach [Keane et al]: 
% Used for: Steerable undercarriage design [pg 96], Constraint Analysis Initial Guidance [pg 127]

Overall aircraft design was initially informed by a text by Keane et al. \cite{keane-17}.
Along with research into general unmanned aerial vehicle (UAV) designs, this gave an insight into the design procedure $-$ and why certain decisions are made during the design process $-$ which allowed a general guide to be followed, avoiding some mistakes and streamlining the design process.
Of particular use was the aspect ratio of the wings used in the designs outlined in the text, along with the wing loadings, as this gave a starting point for the UAV's high level design.
Other useful data were the lift and drag coefficient values, and landing loads were also used when considering landing gear designs.
The constraint analysis carried out in the text gave an overview of the areas of the flight envelope to consider when performing the constraint analysis.
This gave greater confidence that areas of the flight envelope were not omitted that could become limiting factors when selecting the power unit, resulting in a UAV that is either underpowered or uncontrollable.

\subsection{Propulsion locations} \label{sec:introduction:background-research:propulsion-locations}

% Propeller-wing interaction for minimum induced loss. [Ilan Kroo] (Kroo, 1986) 
% Used for: Propulsion layout, Wing design, CFD analysis. 

Research conducted by Kroo \cite{kroo-86} provided a theoretical understanding of how the flow field induced by the rotation of a propeller affects a wing's performance.
This study was conducted for the case of inviscid incompressible flows; the same assumptions were made during the development of the electric modular (ELMO) UAV.
The wing, when interacting with the propeller’s swirl, acts as a turbine stator.
It is shown that the propeller swirl losses are reduced by the wing leading to an increase in net propeller efficiency of 6\%.
Furthermore, the increase in local velocity over the wing reduces the circulation required for a given lift, thus reducing self-induced drag. 

Kroo finds an optimal propeller location along the wingspan using a Fourier series, determining it to be at \pc{40} of the semi-span for a fixed lift distribution.
\pc{40} of the semi-span was therefore chosen as one possible location for the motor mounts on the ELMO UAV.

\subsection{Examples of unconventional layouts} \label{sec:introduction:background-research:examples-of-unconventional-layouts}

There exist a number of aircraft that use unconventional engine layouts, which prove to have their advantages and disadvantages.
The Cessna Skymaster uses both tractor and pusher fuselage mounted engines in order to have the increased thrust of a twin engine aircraft, without the drawback of engine out yaw in the event of an engine failure as is seen for conventional twin-engine aircraft.
The rear-mounted engine, however, is prone to overheating while taxiing due to a lack of cooling airflow.

The Piaggio P.180 Avanti has twin pusher propeller engines, mounted inboard on wings towards the rear of the aircraft, and uses canards at the nose for stability.
This layout reduces cabin noise by having the propellers behind the passengers, but does not take advantage of the wake of the propeller to improve lift as is the case with tractor configurations.

Wingtip mounted propeller aircraft have not been possible with piston engines due to the high mass at the wingtip being undesirable for wing strength and inertia in roll.
With the onset of electric aircraft this is now an option, however, and so aircraft such as the Eviation Alice are using wing tip pusher propellers to provide efficient thrust and reduce wingtip vortices while also using a tail propeller to reduce the yaw effect of a wingtip motor failure.
An example of a tractor configuration is the experimental X-57 Maxwell, which will use wingtip motors to provide thrust in cruise and a distributed propulsion system along the wing to aid low speed performance. 

A study conducted by Pooneh focused on analysing the interaction between propeller and wing for a C130J aircraft with the aid of Kostrel simulation tools \cite{pooneh-18}.
In order to simplify the analysis, only the wing and propeller geometries were modelled.
The Dowty six-bladed R391 propellers were tested in three configurations: inboard, outboard, and combined.
These were then compared to a non-installed control case.
Rotational speeds ranging from \rpm{500} to \rpm{2500} were tested for the propellers in order to assess their influence on the wing performance.

It was found that inboard and outboard installations present similar results, although at low speeds the outboard configuration exhibited lower efficiency.
Although the overall increase in dynamic pressure leads to an increase in both lift and drag, it was shown that such an effect is not uniform.
For a propeller rotating anticlockwise when viewed from the front, the upwash induced on the wing on the left side causes an increase in lift coefficient due to an increase in effective incidence angle, while the right section is subject to a reduction of the positive effects caused by the rise in dynamic pressure.
Furthermore, a combined layout produced better results than the single configuration.
Finally, the presence of the propeller induces a separation and stall delay for the wing downstream. 

\section{Stakeholders} \label{sec:introduction:stakeholders}

\importimage{stakeholder-analysis}{summary of invested partners, members, and consumers.}{Stakeholder analysis}{0.9}

This project was proposed by primary stakeholder and Southampton alumnus Dr. Ewan Kirk, whose vision of the electrification of commercial air travel was the driving force behind the project.
Climate change affects everyone around the world, with the issue only worsening at present.
The airline industry is a large contributor to the issue, so the eventual shift towards electric propulsion is becoming more and more of a necessity.
The proposition of this project and the investment provided by Dr. Kirk is one of many steps on the path to an all-electric future in the commercial airline industry.  

The secondary stakeholder is the University of Southampton.
As a project associated with the University, the outcome is important to the profile and stature of the University as one of the leading engineering universities in the country. 

The second set of stakeholders are those directly working on the project.
In this case these comprise the members of group design project (GDP) group 47, as well as the supervisor Prof. Keith Towell and co-supervisor Dr. Mario Ferraro. 

At the time of writing, the only stakeholders are those stated above.
However, the shift towards electric propulsion is not a new concept in the airline industry, with companies looking into its possibility and feasibility.
As a result, in the future new stakeholders and investors may appear, such as aircraft and engine manufacturers who intend to benefit from electronic propulsion.

\end{document}
