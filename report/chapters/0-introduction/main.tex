\documentclass[../../main.tex]{subfiles}

\begin{document}

\chapter{Introduction} \label{introduction}

Air travel is hugely popular, but it is undeniable that the continued use of fossil fuels in such a demanding and highly used market has negative effects on our planet, and is unsustainable.
By removing some limitations of fossil fuel driven aircraft, electrical aircraft may one day have the edge that necessitates their development and replacement of chemically driven aircraft.
This is the vision of ex-student, Dr. Ewan Kirk, who had the dream to research this in the light of upcoming trends in electrically driven vehicles.
Given the opportunity provided by the aviation industry’s lack of progress in this field, a potentially world changing market opportunity is open. 

The commercial aviation industry lags on the shift toward electric driven transport, due to hardware and battery power density limitations.
By optimising every aspect of an aircrafts design, perhaps outside of the norm seen in conventional aircraft, electrically driven passenger and commercial aircraft may one day become a reality, rather than a gimmick.
Efficiency is paramount in electrical aircraft, this project hopes to begin contributing towards the shift to electric commercial aircraft by providing a highly modular test platform to research the most efficient location for electrical propulsion units, as well as finding other benefits due to the move away from chemical power. 

This report will detail the entire process the team took to begin development on this idea, from the initial research, planning, concepting, design, testing, further design, and what we would have hoped to achieve without the impact of COVID-19, and beyond this university group project, where perhaps this design may have been used on a larger scale to further the drive for electrical commercial aircraft. 

\end{document}
