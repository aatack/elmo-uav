\documentclass[../../main.tex]{subfiles}

\begin{document}

\chapter{Introduction} \label{chapter:introduction}

\section{Motivation} \label{sec:introduction:motivation}

Air travel is hugely popular, but it is undeniable that the continued use of fossil fuels in such a demanding and highly used market has negative effects on our planet, and is unsustainable.
By removing some limitations of fossil fuel driven aircraft, electrical aircraft may one day have the edge that necessitates their development and replacement of chemically driven aircraft.
This is the vision of ex-student, Dr. Ewan Kirk, who had the dream to research this in the light of upcoming trends in electrically driven vehicles.
Given the opportunity provided by the aviation industry’s lack of progress in this field, a potentially world changing market opportunity is open. 

The commercial aviation industry lags on the shift toward electric driven transport, due to hardware and battery power density limitations.
By optimising every aspect of an aircrafts design, perhaps outside of the norm seen in conventional aircraft, electrically driven passenger and commercial aircraft may one day become a reality, rather than a gimmick.
Efficiency is paramount in electrical aircraft, this project hopes to begin contributing towards the shift to electric commercial aircraft by providing a highly modular test platform to research the most efficient location for electrical propulsion units, as well as finding other benefits due to the move away from chemical power. 

This report will detail the entire process the team took to begin development on this idea, from the initial research, planning, concepting, design, testing, further design, and what we would have hoped to achieve without the impact of COVID-19, and beyond this university group project, where perhaps this design may have been used on a larger scale to further the drive for electrical commercial aircraft. 

\section{Background research} \label{sec:introduction:background-research}

Small Unmanned Fixed-Wing Aircraft Design: A Practical Approach [Keane et al]: 

Used for: Steerable undercarriage design [pg 96], Constraint Analysis Initial Guidance [pg 127]  

The Small Unmanned Fixed-Wing Aircraft Design text was used to give a direction at the beginning of the project in terms of the overall aircraft design.
Along with research into general UAV designs, this gave an insight into the design procedure, and why certain decision are made in the design process, which allowed the design procedure to pass more smoothly as we were able to follow a guide in order to avoid making mistakes.
Of particular use were the aspect ratio of the wings used in the designs outlined in the text, along with the wing loadings, as this gave a starting point for the constraint analysis of the design of the ELMO UAV.
Other useful data were the lift and drag coefficient values, and the landing loads were also used when considering landing gear designs.
The constraint analysis carried out in the text gave a guide of the areas of the flight envelope to consider when performing the constraint analysis in order to not miss out a crucial part of the envelope that could be a limiting factor when selecting the power unit, which could result in a UAV that is either underpowered or uncontrollable. 

Propeller-wing interaction for minimum induced loss. [Ilan Kroo] (Kroo, 1986) 

Used for: Propulsion layout, Wing design, CFD analysis. 

The research conducted in Propeller-wing interaction for minimum induced loss provided a theoretical understanding on how the flow field induced by the rotation of a propeller onto a wing affects the performance of these.
This study was conducted for the case of inviscid incompressible flows; hence the same assumptions made during the development of ELMO UAV.
An optimal location along the wingspan for the propeller is found using Fourier series.
Considering a fixed lift distribution, it was found that the optimal position for the reduction of induced drag is at 40\% of the semi-span.
Thus the use of the same position for one of the configurations is to be tested on ELMO UAV.
The wing, when interacting with the propeller’s swirl, acts like a turbine stator.
It is shown that the propeller swirl losses are reduced by the wing leading to increments in net propeller efficiency of 6\%.
Furthermore, the increment in local velocity over the wing reduces the circulation required for a given lift.
Thus, reducing self-induced drag. 

There are a number of aircraft that use more unconventional layouts of engines, which prove to have their advantages and disadvantages.
The Cessna Skymaster uses both tractor and pusher fuselage mounted engines in order to have the increased thrust of a twin engine aircraft, without the drawback of engine out yaw in the event of an engine failure as is seen for conventional twin engine aircraft.
However, the rear mounted engine is prone to overheating while taxiing due to a lack of cooling airflow.
The Piaggio P.180 Avanti uses twin pusher propeller engines mounted inboard on the wings with the wing positioned towards the rear of the aircraft and canards used at the nose for stability.
This layout reduces cabin noise by having the propellers behind the passengers, however, does not take advantage of the wake of the propeller to improve lift as is the case with tractor configurations.
Wingtip mounted propeller aircraft have not been possible with piston engines due to the high mass at the wingtip being undesirable for wing strength and inertia in roll, however with the onset of electric aircraft this is now an option, and so aircraft such as the Eviation Alice are using wing tip pusher propellers to provide efficient thrust and reduce tip vortices while using a tail propeller to reduce the yaw effect of a tip motor failure.
An example of a tractor configuration is the experimental X-57 Maxwell which will use tip motors to provide thrust in cruise and a distributed propulsion system along the wing to aid low speed performance. 

Computational Study of Propeller-Wing Aerodynamic interaction. [Aref Pooneh] (Pooneh, 2018) 

The study conducted in Computational Study of Propeller-Wing Aerodynamic interaction focused on analysing the interaction of propeller and wing for a C130J aircraft with the aid of Kostrel simulation tools.
In order to simplify the analysis, only the wing and propeller geometries were modelled.
The Dowty six-bladed R391 propellers were tested in three configurations: inboard, outboard and combined.
These were then compared to a non-installed control case.
A range from 500 to 2500rpm rotational speed was tested for the propellers in order to assess their influence on the wing performance.
It was found that inboard and outboard installations present similar results, although for the case of low speed the outboard configuration exhibited lower efficiency.
Although the overall increased in dynamic pressure leads to an increase in both lift and drag, it was shown that such effect is not uniform.
For a counter-clockwise rotating propeller, (front view) the upwash induced on the wing on the left side causes an increase in lift coefficient.
Due to an increase in effective incidence angle.
While the right section is subject to a reduction of the positive effects caused by the rise in dynamic pressure.
Furthermore, a combined layout presented better results compared to the single configuration.
Finally, the presence of the propeller induces a separation and stall delay for the wing downstream. 

\section{Stakeholders} \label{sec:introduction:stakeholders}

\importimage{stakeholder-analysis}{summary of invested partners, members, and consumers.}{Stakeholder analysis}{0.9}

The aim of this project is to create a test craft which can be used to find the most efficient configuration of electric propulsion units for the use of electric propulsion in commercial aircrafts in the future.
This project was proposed by Primary stakeholder and Southampton alumni Dr Ewan Kirk, whose vision for the electrification of commercial air travel was the driving force behind the project.
Climate change affects everyone around the world, with the issue only worsening at present.
The airline industry is a large contributor to the issue, so the eventual shift towards electric propulsion is becoming more and more of a necessity.
The proposition of this project and the investment provided by Dr Ewan Kirk is one of many steps on the path to an all-electric future in the commercial airline industry.  

The secondary stakeholder is the University of Southampton.
As a project associated to the university, the outcome is important to the profile and stature of the university as one of the leading Engineering Universities in the country. 

The second set of stakeholders are the employees of a company.
In this case the ‘employees’ are the members of GDP 47 as well as the supervisor Prof. Keith Towell, and co-supervisor Dr Mario Ferraro. 

At the moment the main stakeholders are those stated above.
However, the shift to electric propulsion is not a new concept in the airline industry, with companies looking into its possibility and feasibility.
As a result, in the future new stakeholders may appear, such as aircraft and engine manufacturers who intend to implement electronic propulsion.

\section{Report structure} \label{sec:introduction:report-structure}

% TODO: fill in

\end{document}
