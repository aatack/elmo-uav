\documentclass[../../main.tex]{subfiles}

\begin{document}

\newchapter{Design Specification}{design-specification}

% TODO: disentangle duplicated ideas

\section{Brief} \label{sec:design-specification:brief}

This project aimed to create a configurable, stable, and reliable test platform for determining the effect of the position of propulsion systems on the efficiency of an electric aerial vehicle.
Specific emphasis was placed on isolating the effect of the location of the propellers on the power required to sustain flight at a particular velocity, through mechanisms such as promoting airflow over lifting surfaces or counteracting wingtip vortices, while negating the effect of other variables such as the mass or frontal area by keeping them constant. 

Integral to the success of this project was the construction of a robust UAV, representing a typical and generic aircraft $-$ and as such capable of producing results that are widely applicable $-$ but which can have propulsion units fixed to it in an entirely modular fashion.
This enables the construction of a wide range of flight configurations, altering the characteristics of the propulsion system by changing its position to be between the front and rear of the fuselage; in multiple locations along the span of the wing, including the wingtips; above or below the wing; or in a tractor or pusher configuration. 

By designing a model capable of completely traversing the design space, a number of experiments were to be run which, when combined with results of computational experiments, would yield insight as to the optimal propulsion configuration for similar fixed-wing electric vehicles. 

\section{Requirements} \label{sec:design-specification:requirements}

As this was a new project, it had not inherited any requirements from previous groups. 
The aircraft would not be competing, so had no requirements or constraints derived from competition entry regulations. 
Based on feedback from the project supervisor, a maximum weight of \kg{7} was targetted, along with a maximum wingspan of approximately \metres{2}.
These were mostly for transportability and based on prior experiences \cite{towell-19}.  % TODO: reference properly

The aircraft was also required to have the ability to be flown with propulsion units in different locations on the model in order to compare the performance of each and find the most efficient. 
Beyond this, the design space was open to exploration and interpretation. 

\subsection{Hardware} \label{sec:design-specification:requirements:hardware}

Relaxed requirements meant that few specific hardware requirements bounded the UAV's design.
The general requirements are listed below.

\begin{itemize}
    \item The UAV is to provide a responsive flying platform for testing in accordance with the project objectives.
        This generally means that the UAV must have:
        \begin{itemize}
            \item wings to provide sufficient lift in all expected operating conditions;
            \item a fuselage for structure and housing of electronics and components;
            \item a tailplane to provide stability and control to the aircraft;
            \item control surfaces to provide control in all rotational axes;
            \item one or more propulsion units to facilitate flight, and for testing the project objectives;
            \item landing gear to allow for safe and reliable take-off and landing.
        \end{itemize}
    \item The UAV is to be mechanically reliable and robust.
    \item The UAV to be easily mantainable, allowing for fixes and alterations using simple tools and minimal knowledge.
\end{itemize}

Some further requirements were also defined based on the project objectives.

\begin{itemize}
    \item The UAV should be simple enough that the propulsion systems can be easily altered, and that the project would not be measuring the results of drastic design alterations from the norm, other than those required to examine the project objectives.
    \item The propulsion system should be configurable into a large variety of layouts, with the UAV design being able to fly in any one of these. 
\end{itemize}

\subsection{System} \label{sec:design-specification:requirements:system}

The avionics of the aircraft were meant to fulfil two main goals.

Several requirements were set for the control systems, mostly to allow for analysis and interpretation of results, either in real time or post-flight, or a combination of both. 

\begin{itemize}
    \item The UAV is to have a safe and reliable electrical system to provide power and control to propulsion, avionics and other systems.
        \begin{itemize}
            \item To achieve this, it was quickly noted that batteries would need to be sized appropriately to provide sufficient voltage and current to components for the duration of flight required to complete tests runs. 
            \item The electronic speed controller (ESC) components of the propulsion system would require passive cooling measures such as airflow ducts to prevent overheating and failure. 
            \item The controls of the control surfaces and propulsion would require mapping to an adequate ground control device such as a remote control, including granular control of direct control surface and flap manipulation, control surface trim, throttle control of the propulsion, and potentially the ability to shut down one of two propulsion units to test the UAV’s safety in an asymmetric thrust scenario. 
        \end{itemize}
    \item The UAV should have a GPS tracking module to allow for either real-time or post-flight analysis of GPS telemetry. 
    \item The UAV must have a speed sensor to allow for either real-time or post-flight analysis of speed. 
    \item The control systems should provide the pilot with assistance in flying the aircraft, generally maintaining its stability.
    \item Autopilot functionality is desirable to ensure the aircraft flies straight and level, providing a reliable and repeatable test environment for experiments.
    This would also make sure that the pilot will not be overburdened trying to fly the aircraft and conduct tests simultaneously.
    \item Toggles for flaps (retracted or deployed) and for outboard motors (to prevent excessive yawing moments in the event of an asymmetric engine failure) should be controllable by the pilot.
    \item Finally, the flight systems should either record or transmit any collected data for analysis.
\end{itemize}

\section{Resources} \label{sec:introduction:resources}

Like all GDPs, the project had an initial allocation of £850 to design and build the UAV.
As the project had been proposed by an alumnus, Dr. Kirk pledged an additional £1500 towards the development were it required; this was, however, to be used as a last resort.
Finally, through a successful pitch to the Northrop Grumman 3D printing fund, an additional grant of up to £1000 was secured to be spent on 3D printed parts. 

Various university facilities were also available to the project team, including access to workshops such as the EDMC and TSRL, and specialist tools such as metalworking machinery, a CNC laser cutter, a CNC hot wire foam cutter, and the university 3D printers.
Access was also granted to electronics workshops and equipment, as well as to vital tools and fasteners.

\end{document}
